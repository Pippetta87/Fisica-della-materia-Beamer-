\section{Strumenti e situazioni in MQ}

\begin{frame}{When density of state function is a good approx to actual discrete energy states?}
If states are close enough together:
\begin{align*}
    &n_0^2=\frac{2mL^2E}{\hbar^2}=\frac{3mL^2kT}{\hbar^2}\tag{Def $n_0$ from thermal energy}\\
    &\Delta E=\frac{\hbar^2}{2mL^2}(2n_0-1)\approx\frac{h^2n_0}{mL^2}\tag{energy level separation}\\
    &T_g=\frac{\Delta E}{k}\ll T
\end{align*}
\todo{Modification for ground state  of BE gas in low T limiti: Riedi, thermal physics, sec 10.1}
\end{frame}

\begin{frame}{Densit\'a stati particle in a box and harmonic oscillator}
    $g(E)=\TDy{E}{N}$: numero di stati tra $[E,E+dE]$ (vedi anche $g(\vec{k})$)
\begin{columns}[T]
 \begin{column}{0.55\textwidth}
\begin{align*}
    &E_{1D}=\alpha n^2=\frac{\hbar^2k_n^2}{2m}\\
    &\rho\,dk=(\Delta n_x)=\frac{L}{2\pi}\,dk\\
    &g_{1D}(E)=\frac{1}{2\alpha^{1/2}}E^{1/2},\ g(k)=\frac{L}{2\pi}\\
    &E_{3D}=\alpha n^2=(n_x^2+n_y^2+n_z^2)\alpha,\ n\frac{\lambda}{2}=L\\
    &N=\frac{1}{8}\frac{4\pi}{3}n^3=\frac{\pi n^3}{6}\tag{$\#$ of modes with given E}\\
    &k_xk_yk_z=\frac{(2\pi)^3}{L_xL_yL_z},\ N(k)=\frac{4\pi k^3V}{3(2\pi)^3}\\
    &g_{3D}(E)=\TDy{n}{N}\TDy{E}{n}=\frac{V}{(2\pi)^3}4\pi k^2\TDy{E}{k} 
\end{align*}
 \end{column}
 \begin{column}{0.45\textwidth}
\begin{align*}
    &E_{1D}=\hbar\omega_0(n+\frac{1}{2})\\
    &g_{1D}(E)=\frac{1}{\hbar\omega_0}\\
    &E_{3D}=\hbar\omega_0(n_x+n_y+n_z+\frac{3}{2})\\
    &g_{3D}(E)=\frac{E^2}{2(\hbar\omega_0)^3}\\
    &\to\frac{E^{n-1}}{(n-1)\prod_i^nE_i}
\end{align*}
 \end{column}
 \end{columns}
\end{frame}

\section{Bosoni}

\subsection{Fotoni}

\begin{frame}{I Fotoni}
Bosoni con $S=1$ solo traffsverso, non interagiscono tra loro (eqs. di Maxwell sono lineari): gas perfetto
\end{frame}

\begin{frame}{Corpo nero}
    Il numero di fotoni in equilibrio con pareti ta temperatura T fluttua atitorno a valor medio richiesto da condizione di equilibrio; energia totale dello stato del campo EM $E(\{n_{\vec{k},pol.}\})=\sum_r\hbar\omega n_{\vec{k},pol.}$. Statistica di BE con $\mu=0$ (energia interna non muta con T, V fissi, $\mu=\PDy{N}{F}$), la funzione di partizione:
\begin{align*}
    &Z(T,V)=\sum_{\{n_r\}}\exp{-\beta\sum_rn_r\epsilon_r}=\prod_{r=1}^{\infty}\frac{1}{1-\exp{-\beta\epsilon_r}},\ \ln{Z}=-2\sum_{\vec{k}}\ln{(1-\exp{-\beta\hbar\omega})}\\
    &\exv{n_r}=-\frac{1}{\beta}\PDy{\epsilon_r}{\log{Z}|_{T,V}}=\frac{1}{\beta}\PDof{\epsilon_r}\log{(1-\exp{-\beta\epsilon_r})}=\frac{1}{\exp{\beta\epsilon_r}-1}\\
    &(\Delta n_r)^2=-\beta\PDy{\epsilon_r}{\exv{n_r}}=\exv{n_r}(1+\exv{n_r})
\end{align*}
Conta degli stati con energia con energia $\epsilon$ ($\epsilon=\hbar\omega=pc$, $p=2\pi\frac{\hbar}{\lambda}=\hbar\frac{\omega}{c}$):
\begin{align*}
    &dn_p=2 V\frac{4\pi}{h^3}p^2\,dp=\frac{V8\pi}{h^3}(\frac{\hbar\omega}{c})^2\frac{\hbar\,d\omega}{c}=\frac{V}{\pi^2}\frac{\omega^2\,d\omega}{c^3}\tag{Stati con energia E (2 pol.)}\\
    &dN_{\omega}=\frac{V}{\pi^2c^3}\frac{\omega^2\,d\omega}{\exp{\beta\hbar\omega}-1}\tag{$\#$ di fotoni in $d\omega$: densit\'a stati in $\hbar\omega$ * occupazione media livello}\\
    &u(\omega,T)=\frac{\hbar\omega^3}{\pi^2c^3(\exp{\beta\hbar\omega}-1)}\tag{photon density per unit V}
\end{align*}
\end{frame}

\begin{frame}{Caratteristiche radiazione nera}
    \begin{align*}
        &U=-\PDof{\beta}\ln{Z}=\sum_{\vec{k}}\hbar\omega\exv{n_{\vec{k}}},\ \frac{U}{V}=\intzi{}u(\omega,T)\,d\omega\tag{energia interna e per unit V}\\
        &\TDy{\omega}{u}=0:\ \frac{\hbar\omega_{M}}{kT}=2.822\tag{legge spostamento Wien}\\
        &u(T)=\intzi{}u(\omega,T)\,d\omega=aT^4,\ a=\frac{\pi^2k^4}{15\hbar^3c^3}\tag{Energy density: Stefan-Boltzman}\\
        &u(\omega,T)=\frac{\omega^2}{\pi^2c^2}kT\tag{Basse freq.: Rayleigh-Jeans}\\
        &u(\omega,T)=\frac{\hbar\omega^3}{\pi^2c^3}\exp{-\beta\hbar\omega}\tag{Alte freq.: legge Wien}\\
        &F=-kT\ln{Z}=\frac{VkT}{\pi^2c^3}\frac{1}{\beta^3\hbar^3}\int x^2\ln{(1-\exp{-x})}\,dx=-\frac{1}{3}VT^4\tag{E.L di Helmoholtz}\\
        &F=E-TS\Rightarrow dF=dE-TdS-SdT=-PdV+\dbar Q-TdS-SdT=-PdV-SdT\\
        &S=-\PDy{T}{F}|_V=\frac{4}{3}aVT^3,\ P=-\PDy{V}{F}|_T=\frac{1}{3}aT^4=\frac{1}{3}u(T)
    \end{align*}
\end{frame}


\begin{frame}{Ancora pressione e equazione di stato}
\begin{align*}
    &P=\frac{1}{\beta}\PDof{V}\ln{Z(T,V)}=\frac{1}{\beta}\sum_r\PDy{\epsilon_e}{\ln{Z}}\PDy{\epsilon_r}{V}=\frac{1}{\beta}\sum_r(-\beta\exv{n_r})\PDy{\epsilon_r}{V}\\
    &\epsilon_r=\hbar ck=\hbar c\sqrt{n_x^2+n_y^2+n_z^2}\frac{2\pi}{L}\propto V^{-\frac{1}{3}}\Rightarrow\PDy{V}{\epsilon_r}=-\frac{1}{3}\frac{\epsilon_r}{V}\\
    &P=\frac{1/3}{V}\sum_r\exv{n_r}\epsilon_r=\frac{u(T)}{3}
 \end{align*}
\end{frame}
